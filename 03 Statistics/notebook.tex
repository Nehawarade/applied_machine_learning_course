
% Default to the notebook output style

    


% Inherit from the specified cell style.




    
\documentclass[11pt]{article}

    
    
    \usepackage[T1]{fontenc}
    % Nicer default font (+ math font) than Computer Modern for most use cases
    \usepackage{mathpazo}

    % Basic figure setup, for now with no caption control since it's done
    % automatically by Pandoc (which extracts ![](path) syntax from Markdown).
    \usepackage{graphicx}
    % We will generate all images so they have a width \maxwidth. This means
    % that they will get their normal width if they fit onto the page, but
    % are scaled down if they would overflow the margins.
    \makeatletter
    \def\maxwidth{\ifdim\Gin@nat@width>\linewidth\linewidth
    \else\Gin@nat@width\fi}
    \makeatother
    \let\Oldincludegraphics\includegraphics
    % Set max figure width to be 80% of text width, for now hardcoded.
    \renewcommand{\includegraphics}[1]{\Oldincludegraphics[width=.8\maxwidth]{#1}}
    % Ensure that by default, figures have no caption (until we provide a
    % proper Figure object with a Caption API and a way to capture that
    % in the conversion process - todo).
    \usepackage{caption}
    \DeclareCaptionLabelFormat{nolabel}{}
    \captionsetup{labelformat=nolabel}

    \usepackage{adjustbox} % Used to constrain images to a maximum size 
    \usepackage{xcolor} % Allow colors to be defined
    \usepackage{enumerate} % Needed for markdown enumerations to work
    \usepackage{geometry} % Used to adjust the document margins
    \usepackage{amsmath} % Equations
    \usepackage{amssymb} % Equations
    \usepackage{textcomp} % defines textquotesingle
    % Hack from http://tex.stackexchange.com/a/47451/13684:
    \AtBeginDocument{%
        \def\PYZsq{\textquotesingle}% Upright quotes in Pygmentized code
    }
    \usepackage{upquote} % Upright quotes for verbatim code
    \usepackage{eurosym} % defines \euro
    \usepackage[mathletters]{ucs} % Extended unicode (utf-8) support
    \usepackage[utf8x]{inputenc} % Allow utf-8 characters in the tex document
    \usepackage{fancyvrb} % verbatim replacement that allows latex
    \usepackage{grffile} % extends the file name processing of package graphics 
                         % to support a larger range 
    % The hyperref package gives us a pdf with properly built
    % internal navigation ('pdf bookmarks' for the table of contents,
    % internal cross-reference links, web links for URLs, etc.)
    \usepackage{hyperref}
    \usepackage{longtable} % longtable support required by pandoc >1.10
    \usepackage{booktabs}  % table support for pandoc > 1.12.2
    \usepackage[inline]{enumitem} % IRkernel/repr support (it uses the enumerate* environment)
    \usepackage[normalem]{ulem} % ulem is needed to support strikethroughs (\sout)
                                % normalem makes italics be italics, not underlines
    

    
    
    % Colors for the hyperref package
    \definecolor{urlcolor}{rgb}{0,.145,.698}
    \definecolor{linkcolor}{rgb}{.71,0.21,0.01}
    \definecolor{citecolor}{rgb}{.12,.54,.11}

    % ANSI colors
    \definecolor{ansi-black}{HTML}{3E424D}
    \definecolor{ansi-black-intense}{HTML}{282C36}
    \definecolor{ansi-red}{HTML}{E75C58}
    \definecolor{ansi-red-intense}{HTML}{B22B31}
    \definecolor{ansi-green}{HTML}{00A250}
    \definecolor{ansi-green-intense}{HTML}{007427}
    \definecolor{ansi-yellow}{HTML}{DDB62B}
    \definecolor{ansi-yellow-intense}{HTML}{B27D12}
    \definecolor{ansi-blue}{HTML}{208FFB}
    \definecolor{ansi-blue-intense}{HTML}{0065CA}
    \definecolor{ansi-magenta}{HTML}{D160C4}
    \definecolor{ansi-magenta-intense}{HTML}{A03196}
    \definecolor{ansi-cyan}{HTML}{60C6C8}
    \definecolor{ansi-cyan-intense}{HTML}{258F8F}
    \definecolor{ansi-white}{HTML}{C5C1B4}
    \definecolor{ansi-white-intense}{HTML}{A1A6B2}

    % commands and environments needed by pandoc snippets
    % extracted from the output of `pandoc -s`
    \providecommand{\tightlist}{%
      \setlength{\itemsep}{0pt}\setlength{\parskip}{0pt}}
    \DefineVerbatimEnvironment{Highlighting}{Verbatim}{commandchars=\\\{\}}
    % Add ',fontsize=\small' for more characters per line
    \newenvironment{Shaded}{}{}
    \newcommand{\KeywordTok}[1]{\textcolor[rgb]{0.00,0.44,0.13}{\textbf{{#1}}}}
    \newcommand{\DataTypeTok}[1]{\textcolor[rgb]{0.56,0.13,0.00}{{#1}}}
    \newcommand{\DecValTok}[1]{\textcolor[rgb]{0.25,0.63,0.44}{{#1}}}
    \newcommand{\BaseNTok}[1]{\textcolor[rgb]{0.25,0.63,0.44}{{#1}}}
    \newcommand{\FloatTok}[1]{\textcolor[rgb]{0.25,0.63,0.44}{{#1}}}
    \newcommand{\CharTok}[1]{\textcolor[rgb]{0.25,0.44,0.63}{{#1}}}
    \newcommand{\StringTok}[1]{\textcolor[rgb]{0.25,0.44,0.63}{{#1}}}
    \newcommand{\CommentTok}[1]{\textcolor[rgb]{0.38,0.63,0.69}{\textit{{#1}}}}
    \newcommand{\OtherTok}[1]{\textcolor[rgb]{0.00,0.44,0.13}{{#1}}}
    \newcommand{\AlertTok}[1]{\textcolor[rgb]{1.00,0.00,0.00}{\textbf{{#1}}}}
    \newcommand{\FunctionTok}[1]{\textcolor[rgb]{0.02,0.16,0.49}{{#1}}}
    \newcommand{\RegionMarkerTok}[1]{{#1}}
    \newcommand{\ErrorTok}[1]{\textcolor[rgb]{1.00,0.00,0.00}{\textbf{{#1}}}}
    \newcommand{\NormalTok}[1]{{#1}}
    
    % Additional commands for more recent versions of Pandoc
    \newcommand{\ConstantTok}[1]{\textcolor[rgb]{0.53,0.00,0.00}{{#1}}}
    \newcommand{\SpecialCharTok}[1]{\textcolor[rgb]{0.25,0.44,0.63}{{#1}}}
    \newcommand{\VerbatimStringTok}[1]{\textcolor[rgb]{0.25,0.44,0.63}{{#1}}}
    \newcommand{\SpecialStringTok}[1]{\textcolor[rgb]{0.73,0.40,0.53}{{#1}}}
    \newcommand{\ImportTok}[1]{{#1}}
    \newcommand{\DocumentationTok}[1]{\textcolor[rgb]{0.73,0.13,0.13}{\textit{{#1}}}}
    \newcommand{\AnnotationTok}[1]{\textcolor[rgb]{0.38,0.63,0.69}{\textbf{\textit{{#1}}}}}
    \newcommand{\CommentVarTok}[1]{\textcolor[rgb]{0.38,0.63,0.69}{\textbf{\textit{{#1}}}}}
    \newcommand{\VariableTok}[1]{\textcolor[rgb]{0.10,0.09,0.49}{{#1}}}
    \newcommand{\ControlFlowTok}[1]{\textcolor[rgb]{0.00,0.44,0.13}{\textbf{{#1}}}}
    \newcommand{\OperatorTok}[1]{\textcolor[rgb]{0.40,0.40,0.40}{{#1}}}
    \newcommand{\BuiltInTok}[1]{{#1}}
    \newcommand{\ExtensionTok}[1]{{#1}}
    \newcommand{\PreprocessorTok}[1]{\textcolor[rgb]{0.74,0.48,0.00}{{#1}}}
    \newcommand{\AttributeTok}[1]{\textcolor[rgb]{0.49,0.56,0.16}{{#1}}}
    \newcommand{\InformationTok}[1]{\textcolor[rgb]{0.38,0.63,0.69}{\textbf{\textit{{#1}}}}}
    \newcommand{\WarningTok}[1]{\textcolor[rgb]{0.38,0.63,0.69}{\textbf{\textit{{#1}}}}}
    
    
    % Define a nice break command that doesn't care if a line doesn't already
    % exist.
    \def\br{\hspace*{\fill} \\* }
    % Math Jax compatability definitions
    \def\gt{>}
    \def\lt{<}
    % Document parameters
    \title{Math for Data Science - Statistics}
    
    
    

    % Pygments definitions
    
\makeatletter
\def\PY@reset{\let\PY@it=\relax \let\PY@bf=\relax%
    \let\PY@ul=\relax \let\PY@tc=\relax%
    \let\PY@bc=\relax \let\PY@ff=\relax}
\def\PY@tok#1{\csname PY@tok@#1\endcsname}
\def\PY@toks#1+{\ifx\relax#1\empty\else%
    \PY@tok{#1}\expandafter\PY@toks\fi}
\def\PY@do#1{\PY@bc{\PY@tc{\PY@ul{%
    \PY@it{\PY@bf{\PY@ff{#1}}}}}}}
\def\PY#1#2{\PY@reset\PY@toks#1+\relax+\PY@do{#2}}

\expandafter\def\csname PY@tok@w\endcsname{\def\PY@tc##1{\textcolor[rgb]{0.73,0.73,0.73}{##1}}}
\expandafter\def\csname PY@tok@c\endcsname{\let\PY@it=\textit\def\PY@tc##1{\textcolor[rgb]{0.25,0.50,0.50}{##1}}}
\expandafter\def\csname PY@tok@cp\endcsname{\def\PY@tc##1{\textcolor[rgb]{0.74,0.48,0.00}{##1}}}
\expandafter\def\csname PY@tok@k\endcsname{\let\PY@bf=\textbf\def\PY@tc##1{\textcolor[rgb]{0.00,0.50,0.00}{##1}}}
\expandafter\def\csname PY@tok@kp\endcsname{\def\PY@tc##1{\textcolor[rgb]{0.00,0.50,0.00}{##1}}}
\expandafter\def\csname PY@tok@kt\endcsname{\def\PY@tc##1{\textcolor[rgb]{0.69,0.00,0.25}{##1}}}
\expandafter\def\csname PY@tok@o\endcsname{\def\PY@tc##1{\textcolor[rgb]{0.40,0.40,0.40}{##1}}}
\expandafter\def\csname PY@tok@ow\endcsname{\let\PY@bf=\textbf\def\PY@tc##1{\textcolor[rgb]{0.67,0.13,1.00}{##1}}}
\expandafter\def\csname PY@tok@nb\endcsname{\def\PY@tc##1{\textcolor[rgb]{0.00,0.50,0.00}{##1}}}
\expandafter\def\csname PY@tok@nf\endcsname{\def\PY@tc##1{\textcolor[rgb]{0.00,0.00,1.00}{##1}}}
\expandafter\def\csname PY@tok@nc\endcsname{\let\PY@bf=\textbf\def\PY@tc##1{\textcolor[rgb]{0.00,0.00,1.00}{##1}}}
\expandafter\def\csname PY@tok@nn\endcsname{\let\PY@bf=\textbf\def\PY@tc##1{\textcolor[rgb]{0.00,0.00,1.00}{##1}}}
\expandafter\def\csname PY@tok@ne\endcsname{\let\PY@bf=\textbf\def\PY@tc##1{\textcolor[rgb]{0.82,0.25,0.23}{##1}}}
\expandafter\def\csname PY@tok@nv\endcsname{\def\PY@tc##1{\textcolor[rgb]{0.10,0.09,0.49}{##1}}}
\expandafter\def\csname PY@tok@no\endcsname{\def\PY@tc##1{\textcolor[rgb]{0.53,0.00,0.00}{##1}}}
\expandafter\def\csname PY@tok@nl\endcsname{\def\PY@tc##1{\textcolor[rgb]{0.63,0.63,0.00}{##1}}}
\expandafter\def\csname PY@tok@ni\endcsname{\let\PY@bf=\textbf\def\PY@tc##1{\textcolor[rgb]{0.60,0.60,0.60}{##1}}}
\expandafter\def\csname PY@tok@na\endcsname{\def\PY@tc##1{\textcolor[rgb]{0.49,0.56,0.16}{##1}}}
\expandafter\def\csname PY@tok@nt\endcsname{\let\PY@bf=\textbf\def\PY@tc##1{\textcolor[rgb]{0.00,0.50,0.00}{##1}}}
\expandafter\def\csname PY@tok@nd\endcsname{\def\PY@tc##1{\textcolor[rgb]{0.67,0.13,1.00}{##1}}}
\expandafter\def\csname PY@tok@s\endcsname{\def\PY@tc##1{\textcolor[rgb]{0.73,0.13,0.13}{##1}}}
\expandafter\def\csname PY@tok@sd\endcsname{\let\PY@it=\textit\def\PY@tc##1{\textcolor[rgb]{0.73,0.13,0.13}{##1}}}
\expandafter\def\csname PY@tok@si\endcsname{\let\PY@bf=\textbf\def\PY@tc##1{\textcolor[rgb]{0.73,0.40,0.53}{##1}}}
\expandafter\def\csname PY@tok@se\endcsname{\let\PY@bf=\textbf\def\PY@tc##1{\textcolor[rgb]{0.73,0.40,0.13}{##1}}}
\expandafter\def\csname PY@tok@sr\endcsname{\def\PY@tc##1{\textcolor[rgb]{0.73,0.40,0.53}{##1}}}
\expandafter\def\csname PY@tok@ss\endcsname{\def\PY@tc##1{\textcolor[rgb]{0.10,0.09,0.49}{##1}}}
\expandafter\def\csname PY@tok@sx\endcsname{\def\PY@tc##1{\textcolor[rgb]{0.00,0.50,0.00}{##1}}}
\expandafter\def\csname PY@tok@m\endcsname{\def\PY@tc##1{\textcolor[rgb]{0.40,0.40,0.40}{##1}}}
\expandafter\def\csname PY@tok@gh\endcsname{\let\PY@bf=\textbf\def\PY@tc##1{\textcolor[rgb]{0.00,0.00,0.50}{##1}}}
\expandafter\def\csname PY@tok@gu\endcsname{\let\PY@bf=\textbf\def\PY@tc##1{\textcolor[rgb]{0.50,0.00,0.50}{##1}}}
\expandafter\def\csname PY@tok@gd\endcsname{\def\PY@tc##1{\textcolor[rgb]{0.63,0.00,0.00}{##1}}}
\expandafter\def\csname PY@tok@gi\endcsname{\def\PY@tc##1{\textcolor[rgb]{0.00,0.63,0.00}{##1}}}
\expandafter\def\csname PY@tok@gr\endcsname{\def\PY@tc##1{\textcolor[rgb]{1.00,0.00,0.00}{##1}}}
\expandafter\def\csname PY@tok@ge\endcsname{\let\PY@it=\textit}
\expandafter\def\csname PY@tok@gs\endcsname{\let\PY@bf=\textbf}
\expandafter\def\csname PY@tok@gp\endcsname{\let\PY@bf=\textbf\def\PY@tc##1{\textcolor[rgb]{0.00,0.00,0.50}{##1}}}
\expandafter\def\csname PY@tok@go\endcsname{\def\PY@tc##1{\textcolor[rgb]{0.53,0.53,0.53}{##1}}}
\expandafter\def\csname PY@tok@gt\endcsname{\def\PY@tc##1{\textcolor[rgb]{0.00,0.27,0.87}{##1}}}
\expandafter\def\csname PY@tok@err\endcsname{\def\PY@bc##1{\setlength{\fboxsep}{0pt}\fcolorbox[rgb]{1.00,0.00,0.00}{1,1,1}{\strut ##1}}}
\expandafter\def\csname PY@tok@kc\endcsname{\let\PY@bf=\textbf\def\PY@tc##1{\textcolor[rgb]{0.00,0.50,0.00}{##1}}}
\expandafter\def\csname PY@tok@kd\endcsname{\let\PY@bf=\textbf\def\PY@tc##1{\textcolor[rgb]{0.00,0.50,0.00}{##1}}}
\expandafter\def\csname PY@tok@kn\endcsname{\let\PY@bf=\textbf\def\PY@tc##1{\textcolor[rgb]{0.00,0.50,0.00}{##1}}}
\expandafter\def\csname PY@tok@kr\endcsname{\let\PY@bf=\textbf\def\PY@tc##1{\textcolor[rgb]{0.00,0.50,0.00}{##1}}}
\expandafter\def\csname PY@tok@bp\endcsname{\def\PY@tc##1{\textcolor[rgb]{0.00,0.50,0.00}{##1}}}
\expandafter\def\csname PY@tok@fm\endcsname{\def\PY@tc##1{\textcolor[rgb]{0.00,0.00,1.00}{##1}}}
\expandafter\def\csname PY@tok@vc\endcsname{\def\PY@tc##1{\textcolor[rgb]{0.10,0.09,0.49}{##1}}}
\expandafter\def\csname PY@tok@vg\endcsname{\def\PY@tc##1{\textcolor[rgb]{0.10,0.09,0.49}{##1}}}
\expandafter\def\csname PY@tok@vi\endcsname{\def\PY@tc##1{\textcolor[rgb]{0.10,0.09,0.49}{##1}}}
\expandafter\def\csname PY@tok@vm\endcsname{\def\PY@tc##1{\textcolor[rgb]{0.10,0.09,0.49}{##1}}}
\expandafter\def\csname PY@tok@sa\endcsname{\def\PY@tc##1{\textcolor[rgb]{0.73,0.13,0.13}{##1}}}
\expandafter\def\csname PY@tok@sb\endcsname{\def\PY@tc##1{\textcolor[rgb]{0.73,0.13,0.13}{##1}}}
\expandafter\def\csname PY@tok@sc\endcsname{\def\PY@tc##1{\textcolor[rgb]{0.73,0.13,0.13}{##1}}}
\expandafter\def\csname PY@tok@dl\endcsname{\def\PY@tc##1{\textcolor[rgb]{0.73,0.13,0.13}{##1}}}
\expandafter\def\csname PY@tok@s2\endcsname{\def\PY@tc##1{\textcolor[rgb]{0.73,0.13,0.13}{##1}}}
\expandafter\def\csname PY@tok@sh\endcsname{\def\PY@tc##1{\textcolor[rgb]{0.73,0.13,0.13}{##1}}}
\expandafter\def\csname PY@tok@s1\endcsname{\def\PY@tc##1{\textcolor[rgb]{0.73,0.13,0.13}{##1}}}
\expandafter\def\csname PY@tok@mb\endcsname{\def\PY@tc##1{\textcolor[rgb]{0.40,0.40,0.40}{##1}}}
\expandafter\def\csname PY@tok@mf\endcsname{\def\PY@tc##1{\textcolor[rgb]{0.40,0.40,0.40}{##1}}}
\expandafter\def\csname PY@tok@mh\endcsname{\def\PY@tc##1{\textcolor[rgb]{0.40,0.40,0.40}{##1}}}
\expandafter\def\csname PY@tok@mi\endcsname{\def\PY@tc##1{\textcolor[rgb]{0.40,0.40,0.40}{##1}}}
\expandafter\def\csname PY@tok@il\endcsname{\def\PY@tc##1{\textcolor[rgb]{0.40,0.40,0.40}{##1}}}
\expandafter\def\csname PY@tok@mo\endcsname{\def\PY@tc##1{\textcolor[rgb]{0.40,0.40,0.40}{##1}}}
\expandafter\def\csname PY@tok@ch\endcsname{\let\PY@it=\textit\def\PY@tc##1{\textcolor[rgb]{0.25,0.50,0.50}{##1}}}
\expandafter\def\csname PY@tok@cm\endcsname{\let\PY@it=\textit\def\PY@tc##1{\textcolor[rgb]{0.25,0.50,0.50}{##1}}}
\expandafter\def\csname PY@tok@cpf\endcsname{\let\PY@it=\textit\def\PY@tc##1{\textcolor[rgb]{0.25,0.50,0.50}{##1}}}
\expandafter\def\csname PY@tok@c1\endcsname{\let\PY@it=\textit\def\PY@tc##1{\textcolor[rgb]{0.25,0.50,0.50}{##1}}}
\expandafter\def\csname PY@tok@cs\endcsname{\let\PY@it=\textit\def\PY@tc##1{\textcolor[rgb]{0.25,0.50,0.50}{##1}}}

\def\PYZbs{\char`\\}
\def\PYZus{\char`\_}
\def\PYZob{\char`\{}
\def\PYZcb{\char`\}}
\def\PYZca{\char`\^}
\def\PYZam{\char`\&}
\def\PYZlt{\char`\<}
\def\PYZgt{\char`\>}
\def\PYZsh{\char`\#}
\def\PYZpc{\char`\%}
\def\PYZdl{\char`\$}
\def\PYZhy{\char`\-}
\def\PYZsq{\char`\'}
\def\PYZdq{\char`\"}
\def\PYZti{\char`\~}
% for compatibility with earlier versions
\def\PYZat{@}
\def\PYZlb{[}
\def\PYZrb{]}
\makeatother


    % Exact colors from NB
    \definecolor{incolor}{rgb}{0.0, 0.0, 0.5}
    \definecolor{outcolor}{rgb}{0.545, 0.0, 0.0}



    
    % Prevent overflowing lines due to hard-to-break entities
    \sloppy 
    % Setup hyperref package
    \hypersetup{
      breaklinks=true,  % so long urls are correctly broken across lines
      colorlinks=true,
      urlcolor=urlcolor,
      linkcolor=linkcolor,
      citecolor=citecolor,
      }
    % Slightly bigger margins than the latex defaults
    
    \geometry{verbose,tmargin=1in,bmargin=1in,lmargin=1in,rmargin=1in}
    
    

    \begin{document}
    
    
    \maketitle
    
    

    
    \section{Math for Data Science -
Statistics}\label{math-for-data-science---statistics}

    \subsection{\#\#\# What is Statistic?}\label{what-is-statistic}

    \begin{itemize}
\tightlist
\item
  Dictionary Definition: A fact or piece of data obtained from a study
  of large quantity of numerical data.
\item
  In general sense : The science of collecting, organizing presenting
  and interpreting data.
\item
  Statistics is often categorized into descriptive and inferential
  statistics.
\item
  It uses graphical methods to help makingnumbers visible for
  communicationpurposes.
\item
  It uses analytical methods which providethe math to model and predict
  variation.
\end{itemize}

\textbf{In Nutshell}

    \subsection{\#\#\# Why do we need
Statistics?}\label{why-do-we-need-statistics}

    \begin{itemize}
\tightlist
\item
  To understand data better.
\item
  To make effective decisions.
\end{itemize}

    \subsection{\#\#\# What are various types of
Statistics?}\label{what-are-various-types-of-statistics}

    \textbf{Types of Statistics:} * Descriptive Statistics - collection,
presentation, description of data * Inferential Statistics - making
decisions and drawing conclusions about populations.

    \subsection{\#\#\# What are Descriptive
Statistics?}\label{what-are-descriptive-statistics}

    \begin{itemize}
\tightlist
\item
  Descriptive statistics involves describing, summarizing and organizing
  the data so it can be easily understood.
\item
  Enable us to present the data in a more meaningful way, which allows
  simpler interpretation of the data.
\item
  Typically, there are two general types of statistic that are used to
  describe data:

  \begin{itemize}
  \tightlist
  \item
    \textbf{Measures of central tendency:} these are ways of describing
    the central position of a frequency distribution for a group of
    data.

    \begin{itemize}
    \tightlist
    \item
      To describe this central position we use measures such as
      \textbf{mode, median, and mean.}
    \end{itemize}
  \item
    \textbf{Measures of spread:} these are ways of summarizing a group
    of data by describing how spread out the scores are.

    \begin{itemize}
    \tightlist
    \item
      To describe this spread we use measures such as \textbf{range,
      quartiles, variance and standard deviation.}
    \end{itemize}
  \end{itemize}
\end{itemize}

    \subsection{\#\#\# What are the Measures of Central
Tendency?}\label{what-are-the-measures-of-central-tendency}

    \begin{itemize}
\tightlist
\item
  Used to summarise where the 'centre' of the data is.
\end{itemize}

    \subsection{\#\#\# Measures of Central Tendency - In Layman's
Terms}\label{measures-of-central-tendency---in-laymans-terms}

    

    \begin{Verbatim}[commandchars=\\\{\}]
{\color{incolor}In [{\color{incolor}26}]:} \PY{n}{marks}\PY{o}{=}\PY{p}{[}\PY{l+m+mi}{2}\PY{p}{,}\PY{l+m+mi}{3}\PY{p}{,}\PY{l+m+mi}{4}\PY{p}{,}\PY{l+m+mi}{5}\PY{p}{,}\PY{l+m+mi}{6}\PY{p}{,}\PY{l+m+mi}{7}\PY{p}{,}\PY{l+m+mi}{8}\PY{p}{,}\PY{l+m+mi}{1000}\PY{p}{]}
         \PY{n}{sum\PYZus{}of\PYZus{}marks}\PY{o}{=}\PY{l+m+mi}{0}
         
         \PY{k}{for} \PY{n}{i} \PY{o+ow}{in} \PY{n}{marks}\PY{p}{:}
             \PY{n}{sum\PYZus{}of\PYZus{}marks}\PY{o}{=}\PY{n}{sum\PYZus{}of\PYZus{}marks}\PY{o}{+}\PY{n}{i}
         
         \PY{n+nb}{print}\PY{p}{(}\PY{l+s+s1}{\PYZsq{}}\PY{l+s+s1}{total\PYZus{}marks : }\PY{l+s+s1}{\PYZsq{}}\PY{p}{,}\PY{n}{sum\PYZus{}of\PYZus{}marks}\PY{p}{)}
         \PY{n+nb}{print}\PY{p}{(}\PY{l+s+s1}{\PYZsq{}}\PY{l+s+s1}{no.of students : }\PY{l+s+s1}{\PYZsq{}}\PY{p}{,}\PY{n+nb}{len}\PY{p}{(}\PY{n}{marks}\PY{p}{)}\PY{p}{)}
         \PY{n+nb}{print}\PY{p}{(}\PY{l+s+s1}{\PYZsq{}}\PY{l+s+s1}{mean: }\PY{l+s+s1}{\PYZsq{}}\PY{p}{,}\PY{n}{sum\PYZus{}of\PYZus{}marks}\PY{o}{/}\PY{n+nb}{len}\PY{p}{(}\PY{n}{marks}\PY{p}{)}\PY{p}{)}
\end{Verbatim}


    \begin{Verbatim}[commandchars=\\\{\}]
total\_marks :  1035
no.of students :  8
mean:  129.375

    \end{Verbatim}

    \subsection{\#\#\# Mean}\label{mean}

    \begin{itemize}
\tightlist
\item
  The mean is also called as arithmetic average.
\item
  To calculate the average, or mean, add all values, then divide by the
  number of individuals.
\item
  It is the ``center of mass.''
\item
  Mean gets affected by the extreme values (Outliers).
\end{itemize}

    \subsection{\#\#\# Median}\label{median}

    \begin{itemize}
\tightlist
\item
  The Median (M) is often called the "middle" value and is the value at
  the midpoint of the observations when they are ranked from smallest to
  largest value.
\item
  In an ordered array, the \textbf{median} is the \textbf{middle} number
  i.e., the number that splits the distribution in half.
\item
  The median is not affected by the extreme values (Outliers).
\end{itemize}

\textbf{Steps to get median:} * Arrange the data from smallest to
largest * If n is odd then the median is the single observation in the
center (at the (n+1)/2 position in the ordering) * If n is even then the
median is the average of the two middle observations (at the (n+1)/2
position; i.e., in between)

    \begin{Verbatim}[commandchars=\\\{\}]
{\color{incolor}In [{\color{incolor}33}]:} \PY{n}{salaries}\PY{o}{=}\PY{p}{[}\PY{l+m+mi}{2}\PY{p}{,}\PY{l+m+mi}{2}\PY{p}{,}\PY{l+m+mi}{2}\PY{p}{,}\PY{l+m+mi}{3}\PY{p}{,}\PY{l+m+mi}{3}\PY{p}{,}\PY{l+m+mi}{3}\PY{p}{,}\PY{l+m+mi}{4}\PY{p}{,}\PY{l+m+mi}{4}\PY{p}{,}\PY{l+m+mi}{4}\PY{p}{,}\PY{l+m+mi}{5}\PY{p}{,}\PY{l+m+mi}{5}\PY{p}{,}\PY{l+m+mi}{5}\PY{p}{,}\PY{l+m+mi}{5}\PY{p}{,}\PY{l+m+mi}{6}\PY{p}{,}\PY{l+m+mi}{7}\PY{p}{,}\PY{l+m+mi}{8}\PY{p}{,}\PY{l+m+mi}{1000}\PY{p}{]}
         \PY{n}{salaries}\PY{o}{.}\PY{n}{sort}\PY{p}{(}\PY{p}{)}
         \PY{n+nb}{print}\PY{p}{(}\PY{l+s+s1}{\PYZsq{}}\PY{l+s+s1}{salaries : }\PY{l+s+s1}{\PYZsq{}}\PY{p}{,}\PY{n}{salaries}\PY{p}{)}
         \PY{n+nb}{print}\PY{p}{(}\PY{l+s+s1}{\PYZsq{}}\PY{l+s+s1}{no.of data points : }\PY{l+s+s1}{\PYZsq{}}\PY{p}{,}\PY{n+nb}{len}\PY{p}{(}\PY{n}{salaries}\PY{p}{)}\PY{p}{)}
         \PY{n+nb}{print}\PY{p}{(}\PY{l+s+s1}{\PYZsq{}}\PY{l+s+s1}{mid element index : }\PY{l+s+s1}{\PYZsq{}}\PY{p}{,}\PY{n+nb}{len}\PY{p}{(}\PY{n}{salaries}\PY{p}{)}\PY{o}{/}\PY{l+m+mi}{2}\PY{p}{)}
         \PY{n+nb}{print}\PY{p}{(}\PY{l+s+s1}{\PYZsq{}}\PY{l+s+s1}{median: }\PY{l+s+s1}{\PYZsq{}}\PY{p}{,}\PY{p}{(}\PY{n}{salaries}\PY{p}{[}\PY{l+m+mi}{3}\PY{p}{]}\PY{o}{+}\PY{n}{salaries}\PY{p}{[}\PY{l+m+mi}{4}\PY{p}{]}\PY{p}{)}\PY{o}{/}\PY{l+m+mi}{2}\PY{p}{)}
\end{Verbatim}


    \begin{Verbatim}[commandchars=\\\{\}]
salaries :  [2, 3, 4, 5, 6, 7, 8, 1000]
no.of data points :  8
mid element index :  4.0
median:  5.5

    \end{Verbatim}

    \subsection{\#\#\# Mode}\label{mode}

    \begin{itemize}
\tightlist
\item
  Mode is the value that occurs most often.
\item
  Mode is not affected by extreme values (Outliers).
\item
  Mode is userd for either numerical or categorical data.
\item
  There may be no mode.
\item
  There may be serveral modes (Multi Modal)
\end{itemize}

    \begin{Verbatim}[commandchars=\\\{\}]
{\color{incolor}In [{\color{incolor} }]:} \PY{n}{Data} \PY{n}{Imputation} \PY{o}{\PYZhy{}} \PY{n}{Mean}\PY{p}{,} \PY{n}{Median}\PY{p}{,} \PY{n}{Mode}\PY{p}{,} \PY{n}{SD}
        
        \PY{n}{Student}\PY{p}{:}
        \PY{o}{\PYZhy{}}\PY{o}{\PYZhy{}}\PY{o}{\PYZhy{}}\PY{o}{\PYZhy{}}\PY{o}{\PYZhy{}}\PY{o}{\PYZhy{}}\PY{o}{\PYZhy{}}\PY{o}{\PYZhy{}}
        \PY{n}{Name}\PY{p}{,} \PY{n}{Age}\PY{p}{,} \PY{n}{Gender}\PY{p}{,} \PY{n}{School}\PY{p}{,} \PY{n}{Marks}\PY{p}{,} \PY{n}{No}\PY{o}{.}\PY{n}{Of} \PY{n}{Siblings}
        
        \PY{n}{Quantitative}\PY{p}{:} \PY{n}{Age}\PY{p}{,} \PY{n}{Marks}\PY{p}{,} \PY{n}{No}\PY{o}{.}\PY{n}{of} \PY{n}{Siblings}
            \PY{n}{Descrete} \PY{p}{(}\PY{n}{Whole} \PY{n}{Number}\PY{p}{)}\PY{p}{:} \PY{n}{No}\PY{o}{.}\PY{n}{of} \PY{n}{Siblings} \PY{l+m+mi}{2}
            \PY{n}{Continues} \PY{p}{(}\PY{n}{Real} \PY{n}{Number}\PY{p}{)}\PY{p}{:} \PY{n}{Age} \PY{p}{(}\PY{l+m+mf}{15.5}\PY{p}{)}\PY{p}{,} \PY{n}{Marks} \PY{p}{(}\PY{l+m+mf}{92.4}\PY{p}{)}
            
        \PY{n}{Qualitative}\PY{p}{:} \PY{n}{Name}\PY{p}{,} \PY{n}{Gender}\PY{p}{,} \PY{n}{School}
        \PY{p}{[}\PY{l+m+mi}{1}\PY{p}{,}\PY{l+m+mi}{2}\PY{p}{,}\PY{l+m+mi}{2}\PY{p}{,}\PY{l+m+mi}{2}\PY{p}{,}\PY{l+m+mi}{2}\PY{p}{,}\PY{l+m+mi}{2}\PY{p}{,}\PY{l+m+mi}{2}\PY{p}{,}\PY{l+m+mi}{2}\PY{p}{,}            \PY{p}{,}\PY{l+m+mi}{100}\PY{p}{,}\PY{l+m+mi}{200}\PY{p}{,}\PY{l+m+mi}{300}\PY{p}{]}
\end{Verbatim}


    \subsection{\#\#\# Measure of Central Tendency Most Useful
When?}\label{measure-of-central-tendency-most-useful-when}

    

    \subsection{\#\#\# What are the Measures of
Spread?}\label{what-are-the-measures-of-spread}

    \begin{itemize}
\tightlist
\item
  Measures of Spread tells us how much a data sample is spread out or
  scattered.
\item
  Measures of Spread

  \begin{itemize}
  \tightlist
  \item
    \textbf{Range}
  \item
    \textbf{Inter Quartile Range}
  \item
    \textbf{Standard Deviation}
  \item
    \textbf{Variance}
  \end{itemize}
\end{itemize}

    \subsection{\#\#\# Range}\label{range}

    \begin{itemize}
\tightlist
\item
  The range is the Maximum value minus the minimum value and gives the
  full extent of the range of observations.
\item
  Notice that the range is one number, the difference between the
  Maximum and the minimum.
\end{itemize}

    \subsection{\#\#\# Inter Quartile Range
(IQR)}\label{inter-quartile-range-iqr}

    \begin{itemize}
\tightlist
\item
  The Inter‐Quartile Range (or sometimes Inner‐Quartile Range).
\item
  We will use the notation IQR -- is defined as the difference between
  the 3rd quartile and the 1st quartile.
\item
  Notice that the IQR gives the range of the middle 50\% of the data.
\end{itemize}

    \subsection{\#\#\# Standard Deviation}\label{standard-deviation}

    \begin{itemize}
\tightlist
\item
  Quantify spread of the distribution by measuring how far observations
  are from mean,
\item
  In other terms this measure is based upon the deviations of each value
  from the mean or average value.
\item
  The interpretation of the standard deviation is as the typical or
  average distance between observed data values and the sample mean
\item
  To calculate the standard deviation, we take each observation and
  subtract the sample mean, x‐bar.
\item
  We square each of the differences, and add these squared deviations.
\item
  We divide that total by n -- 1 where n is the number of observations.
\item
  At this stage we have what is called the \textbf{variance, or
  s‐squared.}
\item
  This is sometimes used in statistical methods, however, it is in units
  which are the square of the original units which makes it difficult to
  interpret.
\item
  We take the square root to obtain the final result for the standard
  deviation, s. This value will have the units of the original data and
  have the interpretation of the typical distance an observation differs
  from the sample mean.
\item
  Because each value is weighted equally, the standard deviation is
  influenced by outliers and extreme values.
\item
  In some scenarios, the standard deviation is less reliable than the
  mean in this regard and should be used with caution for highly skewed
  distribution or distributions with extreme outliers. Even moderate
  skewness and relatively mild outliers can have a dramatic impact on
  the standard deviation.
\end{itemize}

\textbf{Variance}

    \textbf{Standard Deviation}

    \subsection{\#\#\# Variance and Standard Deviation of a
Population}\label{variance-and-standard-deviation-of-a-population}

    

    \subsection{\#\#\# What are the key properties of Standard
Deviation?}\label{what-are-the-key-properties-of-standard-deviation}

    \paragraph{\texorpdfstring{Properties of The Standard Deviation
(\emph{s})}{Properties of The Standard Deviation (s)}}\label{properties-of-the-standard-deviation-s}

\begin{itemize}
\tightlist
\item
  Standard Deviation \emph{s} measures the spread about the mean and
  should be used only when mean is choosen as the measure of center.
\item
  Standard Deviation \emph{s=0}, only when there is \textbf{no spread} .
  This happens only when all observations have the same value.
  Otherwise, \emph{s\textgreater{}0}.
\item
  As, the observations become more spread out about their mean, \emph{s}
  gets larger.
\item
  Standard Deviation \emph{s}, is not resistant to outliers. A few
  outliers can make \emph{s} very large
\end{itemize}

    \subsection{\#\#\# What is the Five‐Number
Summary?}\label{what-is-the-fivenumber-summary}

    \paragraph{The Five-Numer Summary:}\label{the-five-numer-summary}

\begin{itemize}
\tightlist
\item
  The five values: minimum, Q1, Median, Q3, and Maximum make up what is
  commonly called ``the five‐number summary''.
\item
  Each of these values represents a measure of position or location.
\item
  As mentioned abobe, the Five-Numer Summary of a set of obsercations
  consists of the smallest observation, the first quartile, the median,
  the third quartile and the largest observation, written in order from
  smallest to largest.
\item
  \textbf{Minimum Q1 Median Q3 Maximum}
\end{itemize}

    \subsection{\#\#\# How to represent Five Number Summary using Box
Plot?}\label{how-to-represent-five-number-summary-using-box-plot}

    

    \subsection{\#\#\# Suspected outliers: how to detect
outliers?}\label{suspected-outliers-how-to-detect-outliers}

\begin{itemize}
\tightlist
\item
  Outliers are troublesome data points, and it is important to be able
  to identify them.
\item
  One way to raise the flag for a suspected outlier is to compare the
  distance from the suspicious data point to the nearest quartile (Q1 or
  Q3).
\item
  We then compare this distance to the interquartile range (distance
  between Q1 and Q3).
\item
  We call an observation a suspected outlier if it falls more than 1.5
  times the size of the interquartile range (IQR) above the first
  quartile or below the third quartile.
\item
  \textbf{This is called the ``1.5 * IQR rule for outliers."}
\end{itemize}

    \subsection{\#\#\# What are Inferential
Statistics?}\label{what-are-inferential-statistics}

    \begin{itemize}
\tightlist
\item
  Inferential statistics is one of the two main branches of statistics.
\item
  Inferential statistics use a random sample of data taken from a
  population to \textbf{describe} and make \textbf{inferences} about the
  \textbf{population.}
\item
  Inferential statistics are valuable when \textbf{examination of each
  member of an entire population is not convenient or possible. }

  \begin{itemize}
  \tightlist
  \item
    For example, to measure the diameter of each nail that is
    manufactured in a mill is impractical.
  \item
    You can measure the diameters of a representative random sample of
    nails.
  \item
    You can use the information from the sample to make generalizations
    about the diameters of all of the nails.
  \end{itemize}
\end{itemize}

    \subsection{\#\#\# Commonly used Terms - Sample,
Population}\label{commonly-used-terms---sample-population}

    

    \subsection{\#\#\# What is Hypothesis?}\label{what-is-hypothesis}

    \begin{itemize}
\tightlist
\item
  Hypothesis is a theoretical statement concerning a certain feature of
  the studied statistical population.
\end{itemize}

    \subsection{\#\#\# What is a test
statistic?}\label{what-is-a-test-statistic}

    \begin{itemize}
\tightlist
\item
  It is a numerical value calculated from our sample which forms a link
  between our sample and the null hypothesis.
\end{itemize}

    \subsection{\#\#\# What is Hypothesis
Testing?}\label{what-is-hypothesis-testing}

    \begin{itemize}
\tightlist
\item
  Hypothesis testing (or significance test): a procedure of assessing
  whether sample data is consistent with statements (hypotheses) made
  about the statistical population.
\item
  Briefly, we make a decision about the hypothesis on the basis of our
  sample data.
\item
  We want to get answers to questions starting typically like these:

  \begin{itemize}
  \tightlist
  \item
    „Is there a difference between\ldots{}''
  \item
    „Is there a relationship between\ldots{}''
  \end{itemize}
\end{itemize}

    \subsection{\#\#\# What are various Types of
Hypotheses?}\label{what-are-various-types-of-hypotheses}

    \begin{itemize}
\tightlist
\item
  There are two kinds of hypothesis:

  \begin{itemize}
  \tightlist
  \item
    H1: the statement we actually want to test

    \begin{itemize}
    \tightlist
    \item
      usually postulates a non-zero difference or relationship (called
      `alternative hypothesis')
    \item
      E.g: „The mean weight of males and females are different.''
    \end{itemize}
  \item
    H0: a statement which usually claims a zero difference or
    relationship against the H1 (called `null hypothesis').

    \begin{itemize}
    \tightlist
    \item
      E.g: „The mean weight of males and females are not different.''
    \end{itemize}
  \end{itemize}
\end{itemize}

    \subsection{\#\#\# Inferential Statistics - What is a confidence
interval?}\label{inferential-statistics---what-is-a-confidence-interval}

    \begin{itemize}
\tightlist
\item
  A confidence interval is simply a way to measure how well your sample
  represents the population you are studying.
\end{itemize}

    \subsection{\#\#\# What are the different types of
Data?}\label{what-are-the-different-types-of-data}

    

    \subsection{\#\#\# What are the different Levels of
Measurement?}\label{what-are-the-different-levels-of-measurement}

    

    \subsection{\#\#\# Skewed Vs. Symmetric
Data}\label{skewed-vs.-symmetric-data}

    

    \subsection{\#\#\# What are the measures of
Symmetry?}\label{what-are-the-measures-of-symmetry}

    \begin{itemize}
\tightlist
\item
  The \textbf{mean} is pulled toward the skew.
\end{itemize}

    \subsection{\#\#\# What are the Common Data
Distributions?}\label{what-are-the-common-data-distributions}

\begin{itemize}
\tightlist
\item
  Discrete

  \begin{itemize}
  \tightlist
  \item
    Symmetric

    \begin{itemize}
    \tightlist
    \item
      Binomial Distribution
    \item
      Uniform Discrete Distribution
    \end{itemize}
  \item
    Asymmetric a.k.a Skewed

    \begin{itemize}
    \tightlist
    \item
      Geometric Distribution
    \item
      Negative Binomial Distribution
    \item
      Hypergeometric Distribution
    \end{itemize}
  \end{itemize}
\item
  Continues

  \begin{itemize}
  \tightlist
  \item
    Symmetric

    \begin{itemize}
    \tightlist
    \item
      Uniform a.k.a Multimodal Distribution
    \item
      Triangular Distribution
    \item
      Normal Distribution
    \item
      Cauchy Distribution
    \end{itemize}
  \item
    Asymmetric or Skewed

    \begin{itemize}
    \tightlist
    \item
      Exponential Distribution
    \item
      Lognormal Gamma Weibull Distribution
    \item
      Minimum Extreme Distribution
    \end{itemize}
  \end{itemize}
\end{itemize}

    

    \subsection{\#\#\# How to perform data
analysis?}\label{how-to-perform-data-analysis}

    \textbf{Guidelines:} * ALWAYS PLOT DATA BEFORE DECIDING ON A NUMERICAL
SUMMARY. * \textbf{How to choose summary statistics?} * Use: 5-number
summary is better than the mean and s.d. for skewed data; * Use mean \&
s.d. for symmetric data.


    % Add a bibliography block to the postdoc
    
    
    
    \end{document}
